\documentclass[a4paper,10pt]{article}
\usepackage[latin1]{inputenc}
\usepackage[french]{babel}
\usepackage[T1]{fontenc}
\usepackage[dvips]{graphicx}

\begin{document}

\section*{Security aspects in OAR}
In OAR2, security and user switching is managed by the ``oardodo'' script.
It is a suid script and is used to launch a command, a terminal or a script with the privileges of a particular user.
When ``oardodo'' is called, it checks the value of an environment variable: OARDO\_BECOME\_USER.
\begin{itemize}
 \item If this variable is empty, ``oardodo'' will execute the command with the privileges of the superuser (root).
 \item Else, this variable contains the name of the user that will be used to execute the command.\\
\end{itemize}

Here are the scripts/modules where ``oardodo'' is called and which user is used during this call:
\begin{enumerate}
 \item oar\_Judas:
	this module is used for logging and notification.
	\begin{itemize}
 		\item user notification: email or command execution.\\OARDO\_BECOME\_USER = user
	\end{itemize}
 \item oarsub:
	this script is used for submitting jobs or reservations.
	\begin{itemize}
 		\item read user script
		\item connection to the job and the remote shell
		\item keys management
		\item job key export
	\end{itemize}
	for all these functions, the user used in the OARDO\_BECOME\_USER variable is the user that submits the job.
 \item pingchecker:
	this module is used to check resources health. Here, the user is root.
 \item oarexec: 
	executed on the first reserved node, oarexec executes the job prologue ant initiate the job.
	\begin{itemize}
 		\item the ``clean'' method kills every oarsub connection process in superuser mode
		\item ``kill\_children'' method kills every child of the process in superuser mode
		\item execution of a passive job in user mode
		\item getting of the user shell in user mode
		\item checkpointing in superuser mode
	\end{itemize}

 \item job\_resource\_manager:
	The job\_resource\_manager script is a perl script that oar server deploys on nodes to manage cpusets, users, job keys, \ldots
	\begin{itemize}
 		\item cpuset creation and clean is executed in superuser mode
	\end{itemize}

 \item oarsh\_shell: 
	shell program used with the oarsh script. It adds its own process in the cpuset and launches the shell or the script of the user.
	\begin{itemize}
 		\item cpuset filling, ``nice'' and display management are executed as root.
		\item TTY login is executed as user.
	\end{itemize}

 \item oarsh:
	oar's ssh wrapper to connect from node to node. It contains all the context variables usefull for this connection.
	\begin{itemize}
 		\item display management and connection with a user job key file are executed as user.
	\end{itemize}

\end{enumerate}

\end{document}
